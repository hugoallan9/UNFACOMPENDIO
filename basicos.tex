\INEchaptercarta{Alumnos en el ciclo básico}{}



\cajita{Inscritos en básicos }{El número de  inscritos en básicos se obtiene a partir del total de los alumnos registrados al treinta de marzo de cada año escolar, sin distinción de la edad y que se matriculan en el ciclo básico.
	
	 En el año 2009 se inscribieron 671,872 alumnos y en el 2014 se inscribieron 764,415, lo cual muestra un crecimiento de 13.8\%.}{Número de inscritos en el ciclo de educación básica}{República de Guatemala, serie histórica, en datos absolutos}{\ \\[0mm]}{Instituto Nacional de Estadística, con datos del Ministerio de Educación}

\cajita{Inscritos en básicos por sexo}{La distribuciòn por sexos de los inscritos en educación básica muestra que el 53.7\% fueron hombres y de mujeres  46.3\% siendo la diferencia de 7.4 puntos porcentuales.}{Distribución de inscritos en el ciclo de educación básica, por sexo}{República de Guatemala, año 2014, en porcentaje}{\ \\[0mm]}{Instituto Nacional de Estadística, con datos del Ministerio de Educación}

\cajita{Inscritos en básicos por grado}{En la gráfica  se observa que el número de inscritos por grado en básico  se concentra en primer grado, con el 41.7\% y donde menos inscritos hay, es en tercer grado con 27.9\%.}{Distribución de inscritos en el ciclo de educación básica, según el grado escolar}{República de Guatemala, año 2014, en porcentaje}{\ \\[0mm]}{Instituto Nacional de Estadística, con datos del Ministerio de Educación}

\cajita{Inscritos en básicos por grupo étnico}{En la gráfica  se observa que del total de inscritos en básico, los no indígenas representan el 75.2\%.}{Distribución de inscritos en el ciclo de educación básica, por grupo étnico}{República de Guatemala, año 2014, en porcentaje}{\ \\[0mm]}{Instituto Nacional de Estadística, con datos del Ministerio de Educación}


\cajita{Inscritos en básicos por sector educativo}{En la gráfica  se observa que del total de inscritos en básico, el 44.3\% están inscritos en el sector público y también figura el sector Cooperativa con 20.5\% de los estudiantes inscritos.}{Distribución de inscritos en el ciclo de educación básica, por sector educativo}{República de Guatemala, año 2014, en porcentaje}{\ \\[0mm]}{Instituto Nacional de Estadística, con datos del Ministerio de Educación}

\cajita{Inscritos en básicos e idioma}{En la gráfica  se observa que del total de inscritos en básico, el 97.5\% reciben clases en idioma español.}{Distribución de inscritos en el ciclo de educación básica, según el idioma en el que reciben clases}{República de Guatemala, año 2014, en porcentaje}{\ \\[0mm]}{Instituto Nacional de Estadística, con datos del Ministerio de Educación}

\cajota{Inscritos en básicos en los departamentos}{El mapa muestra en color celeste los departamentos con la menor cantidad alumnos inscritos en básico, en comparativo departamental. Estos fueron: El Progreso con 9,966, Zacapa 11,296 y Baja Verapaz 12,509.
	
	 Los departamentos con la mayor cantidad des alumnos inscritos en básico fueron: Quetzaltenango 46,920, San Marcos 50,949 y Guatemala 230,778.}{Número de inscritos en el ciclo de educación básica}{Por departamento, año 2014, en datos absolutos}{ }{Instituto Nacional de Estadística, con datos del Ministerio de Educación}




\INEchaptercarta[Indicadores de educación básica]{Indicadores\\ de educación básica}{}


\cajita{Cobertura bruta}{La tasa bruta de cobertura en básico establece una relación entre la inscripción inicial total sin distinción de edad, y la población de 13 a 15 años. 
	
	En el año 2009 fue de 66.7\% en el año 2014 fue de 69.3\%, presentando un crecimiento del 4\%.}{Tasa bruta de cobertura del ciclo de educación básica}{República de Guatemala, serie histórica, en porcentaje}{\ \\[0mm]}{Instituto Nacional de Estadística, con datos del Ministerio de Educación}

\cajita{Cobertura bruta por sexo}{La tasa bruta de cobertura en básico por sexo fue de 73.8\% para hombres y 64.8\% para mujeres.}{Tasa bruta de cobertura del ciclo de educación básica, por sexo}{República de Guatemala, año 2014, en porcentaje}{\ \\[0mm]}{Instituto Nacional de Estadística, con datos del Ministerio de Educación}

\cajota{Cobertura bruta en los departamentos}{El mapa muestra en color celeste los departamentos con las menores tasas brutas de cobertura en básico, que en el 2014 fueron: Huehuetenango  40.5\%, Quiché 40.7\% y Alta Verapaz 42.1\%. 
	
	Los departamentos con las más altas tasas brutas de cobertura en básico: Retalhuleu 86.6\%, El Progreso 86.9\% y Guatemala 111.9\%.}{Tasa bruta de cobertura del ciclo de educación básica}{Por departamento, año 2014, en porcentaje}{}{Instituto Nacional de Estadística, con datos del Ministerio de Educación}

\cajita{Cobertura neta}{La tasa neta de cobertura en básico es la relación que existe entre la parte de la inscripción inicial que se encuentra en la edad escolar de 13 a 15 años y la población de esa misma edad.
	
	 En el año 2009 fue del 40.3\% y en el 2014 fue de 44\%, presentando un crecimiento de 9.4\%.}{Tasa neta de cobertura del ciclo de educación básica}{República de Guatemala, serie histórica, en porcentaje}{\ \\[0mm]}{Instituto Nacional de Estadística, con datos del Ministerio de Educación}

\cajita{Cobertura neta por sexo}{La tasa neta de cobertura en básico por sexo fue de 45.6\% para hombres y 42.5\% para las mujeres.}{Tasa neta de cobertura del ciclo de educación básica, por sexo}{República de Guatemala, año 2014, en porcentaje}{\ \\[0mm]}{Instituto Nacional de Estadística, con datos del Ministerio de Educación}

\cajota{Cobertura neta en los departamentos}{El mapa muestra en color celeste los departamentos con las menores tasas  netas de cobertura en básico que en el 2014 fueron: Alta Verapaz 22.6\%,  Quiché 24.7\% y Huehuetenango 25.7\%.
	
	 Los departamentos con las mayores tasas netas de cobertura en básico fueron: Sacatepéquez 57.7\%, El Progreso 58\% y Guatemala 69.4\%. }{Tasa neta de cobertura del ciclo de educación básica}{Por departamento, año 2014, en porcentaje}{}{Instituto Nacional de Estadística, con datos del Ministerio de Educación}




\cajita{Repitencia}{La tasa de repitencia en básico es la relación que existe entre el número de repitentes y el número de alumnos que en el año  estaban inscritos en el mismo grado.
	
	En el año 2009 el 3.1\% y en el año 2014 fue de 4.5\%, presentando un crecimiento del 48.4\%.}{Tasa de repitencia del ciclo de educación básica}{República de Guatemala, serie histórica, en porcentaje}{\ \\[0mm]}{Instituto Nacional de Estadística, con datos del Ministerio de Educación}

\cajita{Repitencia por sexo}{La tasa de repitencia en básico por sexo fue del 5.5\% para hombres y 3.5\% para las mujeres.}{Tasa de repitencia del ciclo de educación básica, por sexo}{República de Guatemala, año 2014, en porcentaje}{\ \\[0mm]}{Instituto Nacional de Estadística, con datos del Ministerio de Educación}

\cajota{Repitencia en los departamentos}{El mapa muestra en color celeste los departamentos con las menores tasas de repitencia en básico, que en el 2014 fueron: Petén 2.3\%, Jutiapa 2.4\% y Retalhuleu 2.4\%.
	
	 Los departamentos con las mayores tasas de repitencia en básico fueron: Totonicapán 6.3\%, Chimaltenango 6.8\% y Sacatepéquez 8.5\%. El departamento de Guatemala presentó una tasa de repitencia de 5.0\%.}{Tasa de repitencia del ciclo de educación básica}{Por departamento, año 2014, en porcentaje}{}{Instituto Nacional de Estadística, con datos del Ministerio de Educación}






\cajita{Sobre-edad}{La tasa de sobre-edad en básico es la relación que existe entre la cantidad de alumnos inscritos en los diferentes grados de un nivel educativo, con dos o más años de atraso escolar por encima de la edad correspondiente al grado de estudio.
	
	En el año 2009 fue del 34\% y en el año 2014 fue de 28.2\%, que reprsenta una disminución del 17.3\%.}{Tasa de sobre-edad del ciclo de educación básica}{República de Guatemala, serie histórica, en porcentaje}{\ \\[0mm]}{Instituto Nacional de Estadística, con datos del Ministerio de Educación}

\cajita{Sobre-edad por sexo}{La tasa de sobre-edad en básico por sexo en el 2014 fue de 31.4\% para hombres y 24.4\% para las mujeres.}{Tasa de sobre-edad del ciclo de educación básica, por sexo}{República de Guatemala, año 2014, en porcentaje}{\ \\[0mm]}{Instituto Nacional de Estadística, con datos del Ministerio de Educación}

\cajota{Sobre-edad en los departamentos}{El mapa muestra en color celeste los departamentos con las menores tasas de sobre-edad en básico fueron: Chimaltenango 20.3\%, Quetzaltenango 21.9\% y Jutiapa 22.2\%.
	
	 Los departamentos con las mayores tasas  de sobre-edad en básico: Quiché 31.4\% Guatemala 31.7\% y Alta Verapaz 41.6\%.}{Tasa de sobre-edad del ciclo de educación básica}{Por departamento, año 2014, en porcentaje}{}{Instituto Nacional de Estadística, con datos del Ministerio de Educación}





\cajita{Deserción}{La tasa de deserción en básico se refiere a la cantidad de alumnos que no concluyen el ciclo lectivo.
	
	En el año fue 2009 el 8.2\% y en el año 2014 fue de 5.9\%, lo cual representa una disminución del 28.3\%.}{Tasa de deserción del ciclo de educación básica}{República de Guatemala, serie histórica, en porcentaje}{\ \\[0mm]}{Instituto Nacional de Estadística, con datos del Ministerio de Educación}

\cajita{Deserción por sexo}{La tasa de deserción en básico por sexo, representa el 7.1\% para hombres y 4.5\% para las mujeres.}{Tasa de deserción del ciclo de educación básica, por sexo}{República de Guatemala, año 2014, en porcentaje}{\ \\[0mm]}{Instituto Nacional de Estadística, con datos del Ministerio de Educación}

\cajota{Deserción en los departamentos}{El mapa muestra en color celeste los departamentos con las menores tasas de deserción en básicos, que en el 2014 fueron: Jutiapa 4.2\%, Chimaltenango 4.5\% y Quetzaltenango 4.9\%.
	
	 Los departamentos con las mayores tasas  de deserción en básico: Santa Rosa 9.5\%, Retalhuleu 9.6\% e Izabal 10.4\%. En el departamento de Guatemala la tasa de deserción fue de 7.8\%.	}{Tasa de deserción del ciclo de educación básica}{Por departamento, año 2014, en porcentaje}{}{Instituto Nacional de Estadística, con datos del Ministerio de Educación}




\cajita{Aprobación}{La tasa de aprobación en básico se refiere a la cantidad de alumnos que culminaron y aprobaron el ciclo lectivo.
	
	En el año 2009 fue del 68.4\% y en el año 2014 fue de 69.6\%, presentando un crecimiento de 1.7\%.}{Tasa de aprobación del ciclo de educación básica}{República de Guatemala, serie histórica, en porcentaje}{\ \\[0mm]}{Instituto Nacional de Estadística, con datos del Ministerio de Educación}

\cajita{Aprobación por sexo}{La tasa de aprobación en básico por sexo, representa el 66\% para hombres y 73.6\% para las mujeres.}{Tasa de aprobación del ciclo de educación básica, por sexo}{República de Guatemala, año 2014, en porcentaje}{\ \\[0mm]}{Instituto Nacional de Estadística, con datos del Ministerio de Educación}

\cajota{Aprobación en los departamentos}{El mapa muestra en color celeste los departamentos con las menores tasas de aprobación en básico, que en el 2014 fueron: Sacatepéquez 60.8\%, Quetzaltenango 61.5\% y Chimaltenango 63.9\%.
	
	 Los departamentos con las mayores tasas de aprobación en básico fueron: Chiquimula 76.8\% Jalapa 76.9\% y Petén 77.1\%. El departamento de Guatemala presentó una tasa de aprobación de 68.5\%.}{Tasa de aprobación del ciclo de educación básica}{Por departamento, año 2014, en porcentaje}{}{Instituto Nacional de Estadística, con datos del Ministerio de Educación}



