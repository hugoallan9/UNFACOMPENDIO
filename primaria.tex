\INEchaptercarta{Alumnos en el nivel primario}{}



\cajita{Inscritos en primaria }{El número de  inscritos en primaria se obtiene a partir del total de los alumnos que tienen hasta doce años, registrados al treinta de marzo de cada año escolar.  
	
En el año 2009 se inscribieron 2,659,776 alumnos y en el 2014 se inscribieron 2,476,379 alumnos, lo cual muestra una disminución de 6.9\%.}{Número de inscritos en el ciclo de educación primaria}{República de Guatemala, serie histórica, en datos absolutos}{\ \\[0mm]}{Instituto Nacional de Estadística, con datos del Ministerio de Educación}

\cajita{Inscritos en primaria por sexo}{En la presente gráfica se observa que el porcentaje de hombres inscritos en primaria es 51.7\% y de mujeres  48.3\% siendo la diferencia de 3.4 puntos porcentuales.}{Distribución de inscritos en el ciclo de educación primaria, por sexo}{República de Guatemala, año 2014, en porcentaje}{\ \\[0mm]}{Instituto Nacional de Estadística, con datos del Ministerio de Educación}

\cajita{Inscritos en primaria por grado}{La presente gráfica desagrega a los inscritos en educación primaria según el grado de estudio.
	
La mayor cantidad de alumnos se concentra en el primer grado, con el 20.3\% y, según esta relación, el 13\% se inscribió en sexto grado.}{Distribución porcentual de inscritos en el ciclo de educación primaria, según el grado escolar}{República de Guatemala, año 2014, en porcentaje}{\ \\[0mm]}{Instituto Nacional de Estadística, con datos del Ministerio de Educación}

\cajita{Inscritos en primaria por etnia}{En la presente gráfica se observa que del total de inscritos en primaria, los no indígenas representan el 61.4\%.}{Distribución de inscritos en el ciclo de educación primaria,\\ por grupo étnico}{República de Guatemala, año 2014, en porcentaje}{\ \\[0mm]}{Instituto Nacional de Estadística, con datos del Ministerio de Educación}


\cajita{Inscritos en primaria por sector educativo}{En la presente gráfica se observa que del total de inscritos en primaria, el 89.2\% están inscritos en el sector público.}{Distribución de inscritos en el ciclo de educación primaria,\\ por sector educativo}{República de Guatemala, año 2014, en porcentaje}{\ \\[0mm]}{Instituto Nacional de Estadística, con datos del Ministerio de Educación}

\cajita{Inscritos en primaria e idioma}{En la gràfica  se observa que del total de inscritos en primaria, el 81.5\% reciben clases en idioma español.}{Distribución de inscritos en el ciclo de educación primaria, según el idioma en el que reciben clases}{República de Guatemala, año 2014, en porcentaje}{\ \\[0mm]}{Instituto Nacional de Estadística, con datos del Ministerio de Educación}

\cajota{Inscritos en primaria en los departamentos}{El mapa muestra en color celeste los departamentos con la menor cantidad de alumnos inscritos, de esta forma los departamentos donde hubo menos alumnos inscritos en primaria fueron: El Progreso 27,212, Zacapa 38,931 y Sacatepéquez 47,012s. 
	
	 Los departamentos con más alumnos inscritos en el 2014 en primaria fueron: Alta Verapaz 212,762, Huehuetenango 214,366  y Guatemala 432,201 alumnos inscritos.}{Inscritos en el ciclo de educación primaria}{Por departamento, año 2014, en datos absolutos}{}{Instituto Nacional de Estadística, con datos del Ministerio de Educación}




\INEchaptercarta[Indicadores de educación primaria]{Indicadores de \\educación primaria}{}


\cajita{Cobertura bruta}{La tasa bruta de cobertura en primaria presentó en el año 2009 el 118.6\% y en el año 2014 fue de 102.7\%, presentando un decrecimiento del 13.5\%.}{Tasa bruta de cobertura del ciclo de educación primaria}{República de Guatemala, serie histórica, en porcentaje}{\ \\[0mm]}{Instituto Nacional de Estadística, con datos del Ministerio de Educación}

\cajita{Cobertura bruta por sexo}{La tasa bruta de cobertura en primaria por sexo, representa 104.8\% para hombres y 100.4\% para mujeres.}{Tasa bruta de cobertura del ciclo de educación primaria, por sexo}{República de Guatemala, año 2014, en porcentaje}{\ \\[0mm]}{Instituto Nacional de Estadística, con datos del Ministerio de Educación}

\cajota{Cobertura bruta en los departamentos}{El mapa muestra en color celeste los departamentos con las menores tasas brutas de cobertura en primaria, que fueron: Petén 85.7\%, Chimaltenango 90\% y Totonicapán con 90.3\%. 
	
	 Los departamentos con las más altas tassa bruta de cobertura en primaria en el 2014, fueron: Santa Rosa 112.3\%, San Marcos 112.6\% y Retalhuleu 113.4\%. El departamento de Guatemala presentó una tasa de 103.8\%.}{Tasa bruta de cobertura del ciclo de educación primaria}{Por departamento, año 2014, en porcentaje}{}{Instituto Nacional de Estadística, con datos del Ministerio de Educación}

\cajita{Cobertura neta}{La tasa neta de cobertura en primaria presentó el año 2009 el 98.7\% y en el año 2014 fue de 85.4\%, presentando un decrecimiento del 13.5\%.}{Tasa neta de cobertura del ciclo de educación primaria}{República de Guatemala, serie histórica, en porcentaje}{\ \\[0mm]}{Instituto Nacional de Estadística, con datos del Ministerio de Educación}

\cajita{Cobertura neta por sexo}{La tasa neta de cobertura en primaria por sexo, representa el 86\% para hombres y 84.8\% para las mujeres.}{Tasa neta de cobertura del ciclo de educación primaria, por sexo}{República de Guatemala, año 2014, en porcentaje}{\ \\[0mm]}{Instituto Nacional de Estadística, con datos del Ministerio de Educación}

\cajota{Cobertura neta en los departamentos}{El mapa muestra en color celeste los departamentos con las menores tasas netas de cobertura en primaria, que fueron: Petén 68.0\%, Totonicapán 74.2\% y Sololá 75.9.  
	
	Los departamentos donde hubo alta tasa neta de cobertura en primaria fueron: Zacapa 93.2\%, Retalhuleu 93.6 y San Marcos 93.8\% .El departamento de Guatemala presentó una tasa de  91.3\%. }{Tasa neta de cobertura del ciclo de educación primaria}{Por departamento, año 2014, en porcentaje}{}{Instituto Nacional de Estadística, con datos del Ministerio de Educación}




\cajita{Repitencia}{La tasa de repitencia en primaria es la relación que existe entre el número de repitentes y el número de alumnos que en el año  estaban inscritos en el mismo grado.
	
	En el año 2009 fue del 11.5\% y en el 2014 fue de 10.2\%, presentando un decrecimiento del 11.3\%.}{Tasa de repitencia del ciclo de educación primaria}{República de Guatemala, serie histórica, en porcentaje}{\ \\[0mm]}{Instituto Nacional de Estadística, con datos del Ministerio de Educación}

\cajita{Repitencia por sexo}{La tasa de repitencia en primaria por sexo, representa el 11.2\% para hombres y 9.2\% para las mujeres.}{Tasa de repitencia del ciclo de educación primaria, por sexo}{República de Guatemala, año 2014, en porcentaje}{\ \\[0mm]}{Instituto Nacional de Estadística, con datos del Ministerio de Educación}

\cajota{Repitencia en los departamentos}{El mapa muestra en color celeste los departamentos con las menores tasas de repitencia en primaria, que fueron: Guatemala 4.6\%, Jutiapa 7.3\% y Retalhuleu 7.4 \%.  
	
	Los departamentos con las mayores tasas  de repitencia en primaria: Jalapa 13.5\%, Quiché 13.7\% y Alta Verapaz con 16.5\%.}{Tasa de repitencia del ciclo de educación primaria}{Por departamento, año 2014, en porcentaje}{}{Instituto Nacional de Estadística, con datos del Ministerio de Educación}






\cajita{Sobre-edad}{La tasa de  sobre-edad , es la relación que existe entre la cantidad de alumnos inscritos en los diferentes grados de un nivel educativo, con dos o más años de atraso escolar por encima de la edad correspondiente al grado de estudio. 
	
	En el año 2009 fue del 51.7\% y en el año 2014 fue de 20.8\%, presentando un decrecimiento del 59.8\%.}{Tasa de sobre-edad del ciclo de educación primaria}{República de Guatemala, serie histórica, en porcentaje}{\ \\[0mm]}{Instituto Nacional de Estadística, con datos del Ministerio de Educación}

\cajita{Sobre-edad por sexo}{La tasa de  sobre-edad en primaria por sexo, representa el 22.7\% para hombres y 18.7\% para las mujeres.}{Tasa de sobre-edad del ciclo de educación primaria, por sexo}{República de Guatemala, año 2014, en porcentaje}{\ \\[0mm]}{Instituto Nacional de Estadística, con datos del Ministerio de Educación}

\cajota{Sobre-edad en los departamentos}{El mapa muestra en color celeste los departamentos con las menores tasas de sobre-edad en primaria en el 2014, que fueron: Guatemala 10.7\%, Sacatepéquez 12.5\% y Chimaltenango 15.1\%. 
	
	 Los departamentos con las mayores tasas  de sobre-edad en primaria: Quiché 27.0\% Petén 28.9\% y Alta Verapaz 30.9\%.}{Tasa de sobre-edad del ciclo de educación primaria}{Por departamento, año 2014, en porcentaje}{}{Instituto Nacional de Estadística, con datos del Ministerio de Educación}





\cajita{Deserción}{La tasa de deserción en primaria , se refiere a la cantidad de alumnos que no concluyen el ciclo lectivo. 
	
	En el año 2009 fue de 5.5\% y en el año 2014 fue de 3.5\%, presentando un decrecimiento de 3.5\%.}{Tasa de deserción del ciclo de educación primaria}{República de Guatemala, serie histórica, en porcentaje}{\ \\[0mm]}{Instituto Nacional de Estadística, con datos del Ministerio de Educación}

\cajita{Deserción por sexo}{La tasa de deserción en primaria por sexo, representa el 3.8\% para hombres y 3.1\% para las mujeres.}{Tasa de deserción del ciclo de educación primaria, por sexo}{República de Guatemala, año 2014, en porcentaje}{\ \\[0mm]}{Instituto Nacional de Estadística, con datos del Ministerio de Educación}

\cajota{Deserción en los departamentos}{El mapa muestra en color celeste los departamentos con las menores tasas de deserción en primaria: Chimaltenango 1.8\%, Quiché 1.9\% y Sacatepéquez 2\%. 
	
	 Los departamentos con las mayores tasas de deserción en primaria en el 2014, fueron: Izabal 6.6\%, Retalhuleu 6.6 y Petén 7.2\%. En el departamento de Guatemala la tasa de deserción fue de 2.5\%.	}{Tasa de deserción del ciclo de educación primaria}{Por departamento, año 2014, en porcentaje}{}{Instituto Nacional de Estadística, con datos del Ministerio de Educación}




\cajita{Aprobación}{La tasa de aprobación en primaria se refiere a la cantidad de alumnos que culminaron y aprobaron el ciclo lectivo.
	
	En el año 2009 fue de 86.4\% y en el año 2014 fue de 86.6\%, presentando un crecimiento de 0.2\%.}{Tasa de aprobación del ciclo de educación primaria}{República de Guatemala, serie histórica, en porcentaje}{\ \\[0mm]}{Instituto Nacional de Estadística, con datos del Ministerio de Educación}

\cajita{Aprobación por sexo}{La tasa de aprobación en primaria por sexo, representa el 85.3\% para hombres y 87.9\% para las mujeres.}{Tasa de aprobación del ciclo de educación primaria, por sexo}{República de Guatemala, año 2014, en porcentaje}{\ \\[0mm]}{Instituto Nacional de Estadística, con datos del Ministerio de Educación}

\cajota{Aprobación en los departamentos}{El mapa muestra en color celeste los departamentos con las menores tasas de aprobación en primaria: Alta Verapaz 77.1\%, Chiquimula 81.3\% y Quiché 81.8\%.
	
	  Los departamentos con las mayores tasas  de aprobación en primaria fueron: Retalhuleu 88.9\%, Sacatepéquez 89.9\% y Guatemala 93.7\%. }{Tasa de aprobación del ciclo de educación primaria}{Por departamento, año 2014, en porcentaje}{}{Instituto Nacional de Estadística, con datos del Ministerio de Educación}



