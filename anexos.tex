
\appendix




\INEchaptercarta{Aspectos metodológicos}{}


\begin{itemize}
	
	
	\item[\large\textbf{a)}$\ $]	\textbf{\large Recepción de Información:} \\[-3mm]
	
		 La Información utilizada en el presente documento, es referente a la oferta educativa que existe en el país, consistente en educación formal (desde preprimaria hasta diversificado) asimismo como registros obtenidos del Sistema de Encuestas de Hogares (Encuesta Nacional de Empleo e Ingresos – ENEI- y Encuesta Nacional de Condiciones de Vida – ENCOVI-), también se utilizó información generada de los Hechos Vitales, de la Unidad de Salud del Instituto Nacional de Estadística; Datos que se obtuvieron del Ministerio de Finanzas Públicas, a través del SECOIN y Datos del Banco de Guatemala; información de Educación Superior o Universitaria, que se generó a partir de los registros sobre matrícula y graduados de trece universidades, de las catorce que están autorizadas en el país (una universidad estatal y doce privadas).

	\item[\large\textbf{b)}$\ $]	\textbf{\large Generación de indicadores:} \\[-3mm]
	
		 Se utilizaron las bases de datos de estadística inicial del año 2014, sobre educación formal, obtenidas del Ministerio de Educación de Guatemala – MINEDUC-; Con esta información se procesaron los capítulos sobre las tasas netas, brutas, cálculo de sobre edad y demás información referente a los niveles desde preprimaria hasta diversificado.  
		 
		 Bases de datos de la ENEI 2014; se utilizaron las bases de las encuestas de la ENEI para procesar el capitulo referente a alfabetismo,  educación y empleo. 
		
		   ENCOVI 2011, para el cálculo del gasto en educación en los hogares.
		
		 Base de hechos vitales del 2014; información generada por los registros de la Unidad de Estadísticas de Salud del INE, de donde se obtuvo información para realizar el capitulo sobre los nacimientos y las características educativas de las madres.
		
		 Registros administrativos de educación superior; esta información fue obtenida de las universidades siguientes: Universidad de San Carlos de Guatemala, Universidad Mariano Gálvez de Guatemala, Universidad del Valle de Guatemala, Universidad Galileo, Universidad Francisco Marroquín, Universidad Rafael Landivar, Universidad Mesoamericana, Universidad San Pablo de Guatemala, Universidad Rural, Universidad InterNaciones y Universidad Panamericana. De las universidades se obtuvo información para el capitulo sobre matricula y graduaciones generadas por los dos sectores público y privado de educación universitaria.
		
		 También se realizaron consultas a la página del Sistema de Contabilidad Integrada – SICOIN-, del Ministerio de Finanzas Públicas del Gobierno de Guatemala y consulta en la página del Banco de Guatemala, para realizar algunas gráficas referente al gasto de gobierno en educación.
		
	\item[\large\textbf{c)}$\ $]	\textbf{\large Metodología y presentación de resultados:} \\[-3mm]



 Se realizaron las gráficas, en base a una lista de contenido propuesto para cada capítulo, se siguieron los estándares internacionales de estadísticas de educación. 

 El  tema de alfabetismo, a partir de la ENEI, se generó según recomendación del Instituto de Estadística – UIS-, de la UNESCO. Los datos estadísticos de educación que fueron generados de las bases de datos de la ENEI 2014 y ENCOVI 2011, se calcularon con base a todas las respuestas a las boletas de encuesta y no se excluyó ninguna de las respuestas que se tienen en dicha base; se utilizó el factor de expansión para todos los datos generados.

 El tema de educación superior se trabajó basado en las clasificaciones de Nivel educativo, facultades y escuelas, según la Clasificación Internacional Normalizada de la Educación – CINE-, del Instituto de Estadística de la UNESCO y la agrupación de Campos Científicos se realizó en base al Manual de Frascati, de la Organización para la Cooperación y Desarrollo Económicos – OCDE-, que es la propuesta de Norma Práctica para encuestas de Investigación y Desarrollo.


\end{itemize}


