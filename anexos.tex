
\appendix




\INEchaptercarta{Aspectos metodológicos}{}


\begin{itemize}
	
	
	\item[\large\textbf{a)}$\ $]	\textbf{\large Recepción de Información:} \\[-3mm]
	
		 La Información utilizada en el presente documento, es referente a la oferta educativa que existe en el país, consistente en educación formal (desde preprimaria hasta diversificado) asimismo como registros obtenidos del Sistema de Encuestas de Hogares (Encuesta Nacional de Empleo e Ingresos – ENEI- y Encuesta Nacional de Condiciones de Vida – ENCOVI-), también se utilizó información generada de los Hechos Vitales, de la Unidad de Salud del Instituto Nacional de Estadística y Datos del Banco de Guatemala.

	\item[\large\textbf{b)}$\ $]	\textbf{\large Generación de indicadores:} \\[-3mm]
	
		 Se utilizaron las bases de datos de estadística inicial del año 2014, sobre educación formal, obtenidas del Ministerio de Educación de Guatemala – MINEDUC-; Con esta información se procesaron los capítulos sobre las tasas netas, brutas, cálculo de sobre edad y demás información referente a los niveles desde preprimaria hasta diversificado.  
		 
		 Bases de datos de la ENEI 2014; se utilizaron las bases de las encuestas de la ENEI para procesar el capitulo referente a alfabetismo,  educación y empleo. 
		
		   ENCOVI 2014, para el cálculo del gasto en educación en los hogares.
		
		 Base de hechos vitales del 2014; información generada por los registros de la Unidad de Estadísticas de Salud del INE, de donde se obtuvo información para realizar el capitulo sobre los nacimientos y las características educativas de las madres.
		

		
	
		
	\item[\large\textbf{c)}$\ $]	\textbf{\large Metodología y presentación de resultados:} \\[-3mm]



 Se realizaron las gráficas, en base a una lista de contenido propuesto para cada capítulo, se siguieron los estándares internacionales de estadísticas de educación. 

 El  tema de alfabetismo, a partir de la ENEI, se generó según recomendación del Instituto de Estadística – UIS-, de la UNESCO. Los datos estadísticos de educación que fueron generados de las bases de datos de la ENEI 2014 y ENCOVI 2014, se calcularon con base a todas las respuestas a las boletas de encuesta y no se excluyó ninguna de las respuestas que se tienen en dicha base; se utilizó el factor de expansión para todos los datos generados.




\end{itemize}


