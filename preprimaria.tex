\INEchaptercarta{Alumnos en preprimaria}{}



\cajita{Inscritos }{}{Número de inscritos en el ciclo de educación preprimaria}{República de Guatemala, serie histórica, en datos absolutos}{\ \\[0mm]}{Instituto Nacional de Estadística, con datos del Ministerio de Educación}

\cajita{Inscritos por sexo}{}{Distribución de inscritos en el ciclo de\\ educación preprimaria, por sexo}{República de Guatemala, año 2014, en porcentaje}{\ \\[0mm]}{Instituto Nacional de Estadística, con datos del Ministerio de Educación}



\cajita{Inscritos por etnia}{}{República de Guatemala, año 2014, en porcentaje}{\ \\[0mm]}{Instituto Nacional de Estadística, con datos del Ministerio de Educación}


\cajita{Inscritos por sector educativo}{}{Distribución de inscritos en el ciclo de educación preprimaria,\\ por sector educativo}{República de Guatemala, año 2014, en porcentaje}{\ \\[0mm]}{Instituto Nacional de Estadística, con datos del Ministerio de Educación}

\cajita{Inscritos e idioma}{Del total de alumnos inscritos en preprimaria, el 85.2\% recibieron clases en idioma español y el 14.8\% en algún idioma maya.}{Distribución de inscritos en el ciclo de educación preprimaria, según el idioma en el que reciben clases}{República de Guatemala, año 2014, en porcentaje}{\ \\[0mm]}{Instituto Nacional de Estadística, con datos del Ministerio de Educación}

\cajota{Inscritos en los departamentos}{La coloración del mapa indica con color celeste claro los departamentos que tuvieron la menor cantidad de alumnos inscritos en el ciclo de educación preprimaria, los cuales fueron: El progreso con 7,981, Zacapa con 11,255 y Baja Verapaz con 12,292 alumnos inscritos. 
	
	Los departamentos con mayor cantidad de alumnos inscritos en preprimaria fueron: Alta Verapaz 30,620, San Marcos 34,965 y Guatemala 126,620.}{Número de inscritos en el ciclo de educación preprimaria}{Por departamento, año 2014, en datos absolutos}{}{Instituto Nacional de Estadística, con datos del Ministerio de Educación}




\INEchaptercarta{Indicadores de preprimaria}{}


\cajita{Cobertura bruta}{La tasa bruta de cobertura, establece una relación entre la inscripción inicial total sin distinción de edad, y la población menor de siete años.
	
	En el 2009 la tasa bruta de cobertura en preprimaria fue de 72.1\% y en 2014 fue de 63.5\%, presentando una disminución del 11.9\%.}{Tasa bruta de cobertura del ciclo de educación preprimaria}{República de Guatemala, serie histórica, en porcentaje}{\ \\[0mm]}{Instituto Nacional de Estadística, con datos del Ministerio de Educación}

\cajita{Cobertura bruta por sexo}{La tasa bruta de cobertura en preprimaria fue del 63\% en niños y el 64\% en niñas.}{Tasa bruta de cobertura del ciclo de\\ educación preprimaria, por sexo}{República de Guatemala, año 2014, en porcentaje}{\ \\[0mm]}{Instituto Nacional de Estadística, con datos del Ministerio de Educación}

\cajota{Cobertura bruta en los departamentos}{Los departamentos que en el 2014 tuvieron la  menor tasa bruta de cobertura en preprimaria fueron: Quiché con el  36.7\%, Totonicapán 41.1\% y Huehuetenango con 42.2\%.
	
	 Los departamentos que tuvieron las más altas tasas brutas de cobertura en preprimaria fueron: Guatemala 93.6\%, El Progreso 95\% y Zacapa  98.5\%.}{Tasa bruta de cobertura del ciclo de educación preprimaria}{Por departamento, año 2014, en porcentaje}{}{Instituto Nacional de Estadística, con datos del Ministerio de Educación}

\cajita{Cobertura neta}{ La tasa neta, es la relación que existe entre la parte de la inscripción inicial que se encuentra en la edad escolar hasta de 6 años y la población en edad escolar hasta 6 años.
	
	 En el 2009 la tasa neta de cobertura en preprimaria fue de 57.1\% y en el 2014 fue de 46.2\%,que representa una disminución del 19\%.}{Tasa neta de cobertura del ciclo de educación preprimaria}{República de Guatemala, serie histórica, en porcentaje}{\ \\[0mm]}{Instituto Nacional de Estadística, con datos del Ministerio de Educación}

\cajita{Cobertura neta por sexo}{La tasa neta de cobertura en preprimaria por sexo, fue  de 46.1\% en hombres y  46.2\% en mujeres.}{Tasa neta de cobertura del ciclo de educación preprimaria, por sexo}{República de Guatemala, año 2014, en porcentaje}{\ \\[0mm]}{Instituto Nacional de Estadística, con datos del Ministerio de Educación}

\cajota{Cobertura neta en los departamentos}{Los departamentos que tuvieron las mas bajas tasas  de cobertura en preprimaria en el 2014 fueron: Quiché 29.4\%, Totonicapán 32.6\% y Alta Verapaz con 34.8\%. 
	
	 Los departamentos que tuvieron las mayores tasas netas de cobertura en preprimaria fueron: Zacapa 59.5\%, El Progreso 59.7\% y Guatemala 66.4\%.}{Tasa neta de cobertura del ciclo de educación preprimaria}{Por departamento, año 2014, en porcentaje}{}{Instituto Nacional de Estadística, con datos del Ministerio de Educación}


