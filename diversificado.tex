\INEchaptercarta{Alumnos en diversificado}{}



\cajita{Inscritos en diversificado }{El número de  inscritos en diversificado, se obtiene a partir del total de los alumnos registrados al treinta de marzo de cada año escolar, que se inscriben en diversificado sin distinción de la edad
	
	 En el año 2009 se inscribieron 310,778 alumnos y en el año 2014 se inscribieron 395,293 alumnos, lo cual muestra un crecimiento de 27.2\%.}{Número de inscritos en el ciclo de educación diversificada}{República de Guatemala, serie histórica, en datos absolutos}{\ \\[0mm]}{Instituto Nacional de Estadística, con datos del Ministerio de Educación}

\cajita{Inscritos en diversificado por sexo}{En la gráfica  se observa que el porcentaje de hombres inscritos en diversificado fue del 50.1\% y de mujeres  49.9\% siendo la diferencia de 0.1 puntos porcentuales.}{Distribución de inscritos en el ciclo de\\ educación diversificada, por sexo}{República de Guatemala, año 2014, en porcentaje}{\ \\[0mm]}{Instituto Nacional de Estadística, con datos del Ministerio de Educación}

\cajita{Inscritos en diversificado por grado}{En la gráfica  se observa el número de inscritos por grado en diversificado  donde se concentra la mayor cantidad de alumnos fue en cuarto grado con el 40.1\%. El 0.2\% de los inscritos estaban en séptimo grado, siendo este  el que tuvo l menor cantidad.}{Distribución de inscritos en el ciclo de educación diversificada, según el grado escolar}{República de Guatemala, año 2014, en porcentaje}{\ \\[0mm]}{Instituto Nacional de Estadística, con datos del Ministerio de Educación}

\cajita{Inscritos en diversificado por etnia}{En la gráfica  se observa que del total de inscritos en diversificado, los no indígenas representaron el 83\%.}{Distribución de inscritos en el ciclo de educación\\ diversificada, por grupo étnico}{República de Guatemala, año 2014, en porcentaje}{\ \\[0mm]}{Instituto Nacional de Estadística, con datos del Ministerio de Educación}


\cajita{Inscritos en diversificado por sector educativo}{En la gráfica  se observa que del total de inscritos en diversificado, el 69.4\% están inscritos en el sector público y también figura el sector Municipal con 1.3\% de los estudiantes inscritos.}{Distribución de inscritos en el ciclo de educación diversificada,\\ por sector educativo}{República de Guatemala, año 2014, en porcentaje}{\ \\[0mm]}{Instituto Nacional de Estadística, con datos del Ministerio de Educación}

\cajita{Inscritos en diversificado e idioma}{En la gráfica  se observa que del total de inscritos en diversificado, 97.1\% reciben clases en idioma español.}{Distribución de inscritos en el ciclo de educación diversificada, según el idioma en el que reciben clases}{República de Guatemala, año 2014, en porcentaje}{\ \\[0mm]}{Instituto Nacional de Estadística, con datos del Ministerio de Educación}

\cajota{Inscritos en diversificado en los departamentos}{El mapa muestra en color celeste los departamentos con la menor cantidad de alumnos inscritos en diversificado fueron: Totonicapán 4,621, El Progreso 5,298 y Baja Verapaz 5,501.
	
	 Los departamentos con la mayor cantidad de alumnos inscritos en diversificado fueron: San Marcos 24,147, Quetzaltenango 31,481 y Guatemala 126,363.}{Número de inscritos en el ciclo de educación diversificada}{Por departamento, año 2014, en datos absolutos}{ }{Instituto Nacional de Estadística, con datos del Ministerio de Educación}




\INEchaptercarta[Indicadores de educación diversificada]{Indicadores\\ de educación diversificada}{}


\cajita{Cobertura bruta}{La tasa bruta de cobertura en diversificado establece una relación entre la inscripción inicial total sin distinción de edad.
	
	En el año 2009 fue de 33.4\% en el año 2014 fue de 38.7\%, presentando un crecimiento de 15.9\%.}{Tasa bruta de cobertura del ciclo de educación diversificada}{República de Guatemala, serie histórica, en porcentaje}{\ \\[0mm]}{Instituto Nacional de Estadística, con datos del Ministerio de Educación}

\cajita{Cobertura bruta por sexo}{La tasa bruta de cobertura en diversificado por sexo fue del 38.6\% para hombres y 38.8\% para mujeres, se observa una diferencia porcentual de 0.2\%.}{Tasa bruta de cobertura del ciclo de educación diversificada, por sexo}{República de Guatemala, año 2014, en porcentaje}{\ \\[0mm]}{Instituto Nacional de Estadística, con datos del Ministerio de Educación}

\cajota{Cobertura bruta en los departamentos}{El mapa muestra en color celeste los departamentos con las menores tasas brutas de cobertura en diversificado en el 2014 fueron: Totonicapán 15.2\%, Alta Verapaz 20.2\% y Huehuetenango 22.2\%.
	
	 Los departamentos con las mayores tasas brutas de cobertura en diversificado: Retalhuleu 50.4\%, Quetzaltenango 58.7\% y Guatemala 63.8\%.}{Tasa bruta de cobertura del ciclo de educación diversificada}{Por departamento, año 2014, en porcentaje}{ }{Instituto Nacional de Estadística, con datos del Ministerio de Educación}

\cajita{Cobertura neta}{La tasa neta de cobertura en diversificado es la relación que existe entre la parte de la inscripción inicial que se encuentra en la edad escolar de 16 a 19 años y la población en edad escolar de 16 a 19 años.
	
	En el año 2009 el 21.2\% de cobertura y en el 2014 fue de 24.5\%, presentando un crecimiento de 15.5\%.}{Tasa neta de cobertura del ciclo de educación diversificada}{República de Guatemala, serie histórica, en porcentaje}{\ \\[0mm]}{Instituto Nacional de Estadística, con datos del Ministerio de Educación}

\cajita{Cobertura neta por sexo}{La tasa neta de cobertura en diversificado por sexo fue del 24.1\% para hombres y 24.8\% para las mujeres, siendo la diferencia de 0.7 puntos porcentuales.}{Tasa neta de cobertura del ciclo de educación diversificada, por sexo}{República de Guatemala, año 2014, en porcentaje}{\ \\[0mm]}{Instituto Nacional de Estadística, con datos del Ministerio de Educación}

\cajota{Cobertura neta en los departamentos}{El mapa muestra en color celeste los departamentos con las menores tasas netas de cobertura en diversificado en el 2014 fueron: Totonicapán 9\%, Alta Verapaz 10.7\% y Quiché 12.4\%.
	
	 Los departamentos con las más altas tasas netas de cobertura en diversificado fueron: El Progreso 33.1\%, Quetzaltenango 37.1\% y Guatemala 41.5\%. }{Tasa neta de cobertura del ciclo de educación diversificada}{Por departamento, año 2014, en porcentaje}{ }{Instituto Nacional de Estadística, con datos del Ministerio de Educación}




\cajita{Repitencia}{La tasa de repitencia en diversificado es la relación que existe entre el número de repitentes y el número de alumnos que en el año  estaban inscritos en el mismo grado.
	
	En el año 2009 fue  del 1.2\% y en el año 2014 fue de 0.9\%, que representa un decrecimiento de 19\%.}{Tasa de repitencia del ciclo de educación diversificada}{República de Guatemala, serie histórica, en porcentaje}{\ \\[0mm]}{Instituto Nacional de Estadística, con datos del Ministerio de Educación}

\cajita{Repitencia por sexo}{La tasa de repitencia en diversificado por sexo fue del 1.1\% para hombres y 0.8\% para las mujeres. Siendo la diferencia de 0.2 puntos porcentuales.}{Tasa de repitencia del ciclo de educación diversificada, por sexo}{República de Guatemala, año 2014, en porcentaje}{\ \\[0mm]}{Instituto Nacional de Estadística, con datos del Ministerio de Educación}

\cajota{Repitencia en los departamentos}{El mapa muestra en color celeste los departamentos con las menores tasas de repitencia en diversificado fueron: Jutiapa 0.2\%  Escuintla 0.4\% y Petén 0.4\%.
	
	 Los departamentos con las más  altas tasa de repitencia en diversificado en el 2014 fueron: Zacapa 1.4\%, Totonicapán 1.8\% y Sololá 1.8\%. El departamento de Guatemala presentó una tasa de repitencia de 1.1\%.}{Tasa de repitencia del ciclo de educación diversificada}{Por departamento, año 2014, en porcentaje}{ }{Instituto Nacional de Estadística, con datos del Ministerio de Educación}






\cajita{Sobre-edad}{La tasa de sobre-edad en diversificado es la relación que existe entre la cantidad de alumnos inscritos en los diferentes grados de un nivel educativo, con dos o más años de atraso escolar por encima de la edad correspondiente al grado de estudio.
	
	En el año 2009 fue de 31.3\% y en el 2014 fue 28.4\%, presentando un decrecimiento de 9.2\%.}{Tasa de sobre-edad del ciclo de educación diversificada}{República de Guatemala, serie histórica, en porcentaje}{\ \\[0mm]}{Instituto Nacional de Estadística, con datos del Ministerio de Educación}

\cajita{Sobre-edad por sexo}{La tasa de sobre-edad en diversificado por sexo fue 30.6\% para hombres y 26.2\% para mujeres. Presentando una diferencia de 4.4 puntos porcentuales.}{Tasa de sobre-edad del ciclo de educación diversificada, por sexo}{República de Guatemala, año 2014, en porcentaje}{\ \\[0mm]}{Instituto Nacional de Estadística, con datos del Ministerio de Educación}

\cajota{Sobre-edad en los departamentos}{El mapa muestra en color celeste los departamentos con las menores tasas de sobre-edad en diversificado: Zacapa 21.1\%, Jutiapa 21.7\% y El Progreso 22.3\%.
	
	 Los departamentos que en el 2014 tuvieron las más altas tasas  de sobre-edad en diversificado: Petén 32.4\%, Quiché 37.7\% y Alta Verapaz 42.6\%. En el departamento de Guatemala, la tasa de sobre-edad fue de 28.3\%. }{Tasa de sobre-edad del ciclo de educación diversificada}{Por departamento, año 2014, en porcentaje}{ }{Instituto Nacional de Estadística, con datos del Ministerio de Educación}





\cajita{Deserción}{La tasa de deserción en diversificado se refiere a la cantidad de alumnos que no concluyen el ciclo lectivo.
	
	Presentó en el año 2009 el 6.5\% y en el año 2014 fue de 1.9\%, presentando un decrecimiento de 70.4\%.}{Tasa de deserción del ciclo de educación diversificada}{República de Guatemala, serie histórica, en porcentaje}{\ \\[0mm]}{Instituto Nacional de Estadística, con datos del Ministerio de Educación}

\cajita{Deserción por sexo}{La tasa de deserción en diversificado por sexo fue del 2.8\% para hombres y 1\% para las mujeres, esto es una diferencia de 1.8 puntos porcentuales.}{Tasa de deserción del ciclo de educación diversificada, por sexo}{República de Guatemala, año 2014, en porcentaje}{\ \\[0mm]}{Instituto Nacional de Estadística, con datos del Ministerio de Educación}

\cajota{Deserción en los departamentos}{Los departamentos donde hubo alta tasa de deserción en diversificado fueron: El Progreso 4.6\%, Zacapa 5.5\% y Jalapa 5.8\%. En el departamento de Guatemala la tasa de deserción fue de 3.4\%.	}{Tasa de deserción del ciclo de educación diversificada}{Por departamento, año 2014, en porcentaje}{ }{Instituto Nacional de Estadística, con datos del Ministerio de Educación}




\cajita{Aprobación}{La tasa de aprobación en diversificado se refiere a la cantidad de alumnos que culminaron y aprobaron el ciclo lectivo.
	
	Presentó el año 2009 el 76\% y en el  2014 fue de 80.1\%, que es una disminución del 19.9\%.}{Tasa de aprobación del ciclo de educación diversificada}{República de Guatemala, serie histórica, en porcentaje}{\ \\[0mm]}{Instituto Nacional de Estadística, con datos del Ministerio de Educación}

\cajita{Aprobación por sexo}{La tasa de aprobación en diversificado por sexo fue del 77.4\% para hombres y 82.7\% para las mujeres.}{Tasa de aprobación del ciclo de educación diversificada, por sexo}{República de Guatemala, año 2014, en porcentaje}{\ \\[0mm]}{Instituto Nacional de Estadística, con datos del Ministerio de Educación}

\cajota{Aprobación en los departamentos}{El mapa muestra en color celeste los departamentos con las menores tasas de aprobación en diversificado en el 2014 fueron: Sacatepéquez 60.8\%, Quetzaltenango 61.5\% y Chimaltenango 63.9\%.
	
	 Los departamentos con las mayores tasas de aprobación en diversificado fueron: Chiquimula 78.8\% Jalapa 78.9\% y Petén 77.1\%. El departamento de Guatemala presentó una tasa de aprobación de 68.5\%.}{Tasa de aprobación del ciclo de educación diversificada}{Por departamento, año 2014, en porcentaje}{ }{Instituto Nacional de Estadística, con datos del Ministerio de Educación}



